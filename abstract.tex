\thispagestyle{plain}

When developing Project management backends, one often must plan for data types, ways to process the data and actions to take. This thesis explores how OOP methodologies benefit project management systems and software when the building blocks of the software or system are represented in terms of “self-sufficient” objects.

To improve the ease and speed with which applications are developed, there is the need for pre-developed libraries or framework. As such, a lot is abstracted from the library consumer, and they can come up with solutions quicker. A project management application is no exception to this and can benefit from a library that already implements or provides an interface for most project management objects and processes to be represented in the form of objects.

This thesis will produce a python library that models project management objects and processes that will be essential when developing a Python-based thesis project management application. For example, in a thesis setting, objects can be created to represent the project charter, the student involved, the professor and any other external stakeholders etc. These building blocks can then be aggregated, according to requirements, to form the core of the given project which should be self-sufficient (i.e., implement all the methods and attributes necessary). The benefits this will provide are many folds from appropriating the benefits of \acs{oop} to \acs{pm} to reduced efforts in the development process due to code reusability to increased flexibility due to modularity and polymorphism.